\section{Software}
I dette afsnit beskrives al det software som er blevet skrevet til microcontrolleren til at styrer de forskellige interne moduler samt de eksterne sensorer vi har sat på bilen. Selve styringen af hvordan bilen håndterer banen findes i sektion \ref{sec:AI}.

\subsection{Generel struktur}
I dette afsnit beskrives der, hvordan koden er struktureret, og hvilke tanker der er blevet gjort omkring opsætningen af koden inden den blev skrevet.

\subsubsection{Retningslinjer}
Da vi gik i gang med at skrive kode, valgte vi lige at bruge lidt tid på at sætte os ned og opsætte nogle retningslinjer for, hvordan kodes skulle skrives. Dette gjorde vi for at få en konsekvent og let læselig kode. Samtidig med at debugging af eventuelle fejl skulle være så nem som mulig. Følgende kriterier blev opsat:

\begin{itemize}[noitemsep,nolistsep,topsep=0ex]
	\item Main filen skal være så "ren" som muligt dvs. at den skal bruges som et samlingspunkt for al koden og ikke som et 		sted, hvor der reelt kører kode.
	\item Alle funktionaliteter skal skrives ind i separate asm filer, som så inkluderes i Main filen.
	\item Alle funktionaliteter indeholder en init macro i formatet "funktionalitet\_Init", som står for at initialiserer 			alt som funktionen skal bruge. 
	\item Alle init macroer "kaldes" fra "setup.asm", som er den sidste inkludering i main filen.
	\item Alle labels og definitioner skal have prefix efter funktionaliteten. Så der ikke overstår brug af de samme labels 		på tværs af asm filer.
	\item Alle interrupt vektorer defineres i en seperat "Interrupt\_Mapping.asm" fil, så det er nemt og overskueligt at 			holde styr på
	\item Alle definitioner af SRAM adresser foretages i filen "SRAM\_Mapping.asm" så der altid er styr på, hvad der er 				tildelt hvilke SRAM-adresser.
\end{itemize}

Ud fra disse definitioner er al kode mere eller mindre skrevet. Hvilket har gjort det meget læseligt og nemt at debugge.

\subsubsection{Eksempler på udførsel}
Figur \ref{fig:MainExample} viser et eksempel på, hvordan alle moduler er skrevet ud i individuelle asm-filer, og inkluderet ind i main filen. Dette gør, at det er nemt at slå et modul fra, hvis man skal teste noget og holde styr på, hvad der står hvor. Figur \ref{fig:SRAMExample} viser, hvordan alle SRAM-definitioner er skrevet ind i den samme asm-fil så der ikke er nogen der kommer til at definerer de samme adresser til forskellige ting. Dette ville nemlig være en svær fejl at opdage, da compileren ikke vil smide en fejl på det. 

\begin{figure}[h]
	\centering
  	\begin{minipage}{.5\textwidth}
  		\centering
  		\begin{minipage}{.85\textwidth}
  			\centering	
			\includegraphics[scale=0.2,width=\linewidth]{Billeder/main_example.PNG}
			\caption{Viser hvordan alle moduler in individuelt inkluderet i main filen}
			\label{fig:MainExample}
		\end{minipage}
	\end{minipage}%
	\begin{minipage}{.5\textwidth}
			\centering
			\begin{minipage}{.85\textwidth}
			\includegraphics[scale=0.1,width=\linewidth]{Billeder/SRAM_Example.PNG}
			\caption{Viser hvordan alle SRAM adresser er defineret i den samme fil for at kunne holde styr på, hvad der er 					brugt.}
			\label{fig:SRAMExample}
			\end{minipage}
	\end{minipage}
\end{figure}


\subsection{Underbyggende funktionaliteter}
I dette afsnit beskrives nogle af de underbyggende funktionaliteter, der er inkorporeret i softwaren. Dette er ting, som ikke er programmerings mæssigt store nok, til at de kræver et underafsnit for sig selv, men som alligevel har dannet grundlag for nogle af de andre funktioner i koden.

\subsubsection{Time}
\textit{Forefindes i filen "Time.asm"}\\
Time biblioteket blev lavet for at have en måde hvorpå bilen kan holde øje med, hvor lang tid der er gået, fra den er sat på banen. Dette bliver brugt flere stedet i koden til f.eks. at sende ud, hvor lang tid en omgang har taget, eller hvis der skal inkorporeres et delay, der ikke opholder bilen, imens den venter. 

Det består af en init macro og en interrupt funktion. Init macroen sætter Timer0 op, til at kører i CTC-mode med en prescaler på 64. Dette giver Timer0 en clock på $\dfrac{16.000.000Hz}{64} = 250.000Hz$ som svarer til at den tæller en op hvert $4\mu s$. Hvis der ikke står nogen værdi i OCR0 registret, så vil timeren også sætte et interrupt, hver gang at den tæller op. Ved at sætte OCR0 registret til en værdi kan man styrer, hvor mange gange Timer0 skal tælle op, før den sætter et interrupt, og begynder forfra med at tælle. Ved at sætte OCR0 til 249 så opnås der et interrupt, hvert milisekund. Hvis man siger $4\mu s  \cdot 249$ så får man dog kun $0.996 ms$, men timeren bruger lige en extra cycle, når der er overflow, og dette giver så i sidste ende en værdi på 250, som ganget sammen med cycletiden bliver $4\mu s  \cdot 250 = 1ms$ \par
Selve interruptet bliver håndteret af funktionen "Timer0\_Update", som sørger for at opdaterer en 24-bit værdi i SRAM, der indeholder tiden i ms, fra bilen blev sat på banen til der, hvor man læser værdien. 24-bit giver en maksimal tid på $2^{24} ms \approx 4,6 timer$ hvilket er forholdsvis lang tid, men en 16-bit værdi ville kun give en maksimal tid på $2^{16}ms \approx 1 minut$, hvilket den meget hurtigt vil kunne opnå, hvis der køres en masse runder for at teste bilen af.

\subsubsection{Delay}
\textit{Forefindes i filen "Delays.asm"}\\
Dette bibliotek indeholder en delay funktion, "Delay\_MS", som kan bruges til at indsætte et delay i en funktion uden at opholde bilen, mens der ventes. Det vil sige, at interrupts stadigvæk kan fungerer, som normalt imens delayet kører. Dette gøres ved at sætte et timestamp der repræsenterer tiden, hvor delayet blev aktiveret. Og så derefter loope indtil at forskellen mellem det timestamp og den nuværende tid er lig med det delay der ønskes. Imens der loopes er det stadigvæk muligt for interrupts at blive aktiveret. Denne funktion bruges bl.a. til at vente et vidst stykke tid, med at starte bilens AI fra den er blevet sat på banen

\subsubsection{I2C}
\textit{Forefindes i filen "I2C.asm"}\\
Dette bibliotek indeholder en række funktioner, der forsøger at gøre I2C/TWI- kommunikation meget mere overskueligt. Der er en init macro der sørger for at starte TWI-modulet op og sætte det op til at kører med en brugervalgt prescaler og TWBR-værdi som man kan tilføje som argument til macroen. Dette giver en mulighed for hurtigt at sætte den frekvens, modulet skal kører ved. Resten af funktionerne i filen er bare kald der gør ens kode meget mere overskuelig. I stedet for at skrive "$LDI R16,(1<<TWINT)|(1<<TWSTA)|(1<<TWEN)$" og loade det ud til modulet for derefter at vente på at det sender en start kommando, så kan der blot skrives "call I2C\_Start" og så bliver den korrekte værdi sendt til modulet, og loopet køres. Det samme gør sig gældende for alle andre I2C-kommandoer såsom Stop, Read og Write. Denne implementation, gør dog ikke brug af fejlkoder, og antager at modulet, sender kommandoerne uden fejl, hvilket det burde gøre i en single-master situation.

\subsubsection{USART}
\textit{Forefindes i filen "USART\_Library.asm"}\\
Dette bibliotek står for at implementerer alle funktioner, der har med at sende kommunikation over USART at gøre. Der er en init macro der sørger for at tænde USART-modulet og sætte dataformatet op som 8 data-bits, 1 stop-bit, no parity. Derudover kan man give UBBRH og UBBRL med som argument som baudraten så bliver sat efter. De fleste funktioner i biblioteket står primært for at sende forskellige data formater over serial i et menneskeligt læseligt format. Såsom funktionen "Decimal\_S16" der kan tage en 16-bit signered værdi og skrive det ud som en serie af ASCII-karakterer. Eller funktionen "Binary" der kan tage en 8-bit værdi og sende det ud som en serie af ASCII 0-og 1-taller. Håndteringen af modtagen data over USART foretages via et interrupt i kommunikations protokollen.

\subsection{Kommunikations Protokol}
I oplægget er der specificeret en kommunikationsprotokol, som bilen skal overholde. Protokollen fungerer ved hjælp af telegrammer. I afsnittet "Beskrivelse af telegram" forklares der, hvad et telegram er. Og i afsnittet "Implementering" beskrives der, hvordan protokollen kodemæssigt er implementeret i microcontrolleren.

\subsubsection{Beskrivelse af telegram}
Protokollen er baseret på "telegrammer" bestående af 3 bytes der henholdsvis repræsenterer en type, kommando og et parameter.\\
Af typer er der specificeret 3 forskellige:
\begin{itemize}[noitemsep,nolistsep,topsep=0ex]
	\item SET - 0x55 - Bruges til at sætte en værdi i bilen
	\item GET - 0xAA - Bruges til at hente en værdi fra bilen
	\item REPLY - 0xBB - Bruges til når der returneres en værdi, som svar på et get/set telegram
\end{itemize}
Det har ikke været nødvendigt at implementerer yderligere typer, så dette er også de eneste tre typer, som implementeringen gør brug af.\\
Af kommandoer er der fra oplæggets side sat et krav om to som bilen som minimum skal overholde, disse er som følger:
\begin{itemize}[noitemsep,nolistsep,topsep=0ex]
	\item Start - 0x10 - Bruges til at starte bilen med en procentdel af maxspændingen ud 			fra parametret
	\item Stop - 0x11 - Bruges til at stoppe bilen med
\end{itemize}

Ud over dette har vi specificeret yderligerer to set kommandoer:
\begin{itemize}[noitemsep,nolistsep,topsep=0ex]
	\item Speed\_H - 0x12 - Bruges til at sætte de øverste 8 bit af en 16-bit værdi, der essentielt bestemmer hvilken hastighed, hastighedskontrollen forsøger at holde
	\item Speed\_L - 0x13 - Bruges til at sætte de nederste 8 bit af en 16-bit værdi, der essentielt bestemmer hvilken hastighed, hastighedskontrollen forsøger at holde
\end{itemize}

Disse to kommandoer er essentielt en kommando, men skal sendes som to seperate kommandoer, da et telegram kun kan bestå af en type, kommando og et parameter. De er blevet brugt meget til at teste hastighedskontrollen af og fastsætte max hastigheder rundt i forskellige sving.\\

Fra oplæggets side er der ikke specificeret nogle get kommandoer, men vi har implementeret de følgende:
\begin{itemize}[noitemsep,nolistsep,topsep=0ex]
	\item Xaccel\_H - 0xA1 - Returner de øverste 8 bit af den signerede 16-bit rå værdi fra acceleration af x-aksen på sensoren
	\item Xaccel\_L - 0xA2 - Returner de nederste 8 bit af den signerede 16-bit rå værdi fra acceleration af x-aksen på sensoren
	\item Zgyro\_H - 0xA3 - Returner de øverste 8 bit af den signerede 16-bit rå værdi af vinkelhastigheden om z-aksen på sensoren
	\item Zgyro\_L - 0xA4 - Returner de nederste 8 bit af den signerede 16-bit rå værdi af vinkelhastigheden om z-aksen på sensoren
	\item Ticks\_H - 0xA5 - Returner de øverste 8 bit af den 16-bit værdi der fortæller, hvor mange "ticks" bilen har kørt på daværende tidspunkt
	\item Ticks\_L - 0xA6 - Returner de nederste 8 bit af den 16-bit værdi der fortæller, hvor mange "ticks" bilen har kørt på daværende tidspunkt
	\item LapTime\_H - 0xA7 - Returner de øverste 8 bit af den 16-bit værdi der fortæller, hvor mange millisekunder banen tog at gennemfører
	\item LapTime\_L - 0xA8 - Returner de nederste 8 bit af den 16-bit værdi der fortæller, hvor mange millisekunder banen tog at gennemfører
	\item LapTicks\_H - 0xA9 - Returner de nederste 8 bit af den 16-bit værdi der fortæller, hvor mange "ticks" der har været på en omgang af banen
	\item LapTicks\_L - 0xAA - Returner de nederste 8 bit af den 16-bit værdi der fortæller, hvor mange "ticks" der har været på en omgang af banen
	\item Data - 0xAB - Bliver brugt til at få bilen til, at returnerer flere data-værdier, uden at skulle sende et get telegram for hver enkelt. Justeres ind alt efter hvad, man gerne vil have tilbage.
	
\end{itemize}

De fleste af kommandoerne består af to telegrammer da mange af værdierne, som er interessante at hente ud af bilen er i et 16-bit format.\\
Til sidst i telegrammet sendes der altid et parameter med, også selvom en funktion ikke gør brug af et parameter.

\subsubsection{Implementering}
Implementeringen kan deles op i to dele, en del der står for at behandle de modtagne bytes. Og en del der står for at eksekverer kommandoerne, når et helt telegram er modtaget. 

\paragraph{Modtagelse}
\textit{Figur \ref{fig:Comm_Modtagelse} Beskriver modtagelsen som et flowchart.} \\

Når USART modulet modtager en byte, vil der blive kaldt et interrupt, der springer til plads 0x1A, hvor kommandoen "jmp Comm\_Received" står. Dette er indgangen til modtagelsesdelen af implementeringen. \\
Først indlæses værdien Byte\_Num fra SRAM. Den holder styr på, hvilken af de 3 bytes der er modtaget. Den første byte, altså typen, har værdien et. Ud fra denne værdi springer programmet så til håndteringen af en modtaget type, modtaget kommando eller en eksekvering af det modtagende telegram. Håndteringen af typer og kommandoer er den sammen. Den modtagende type/kommando holdes op mod de typer/kommandoer, som microcontrolleren kender, og hvis det er en kendt type/kommando, så gemmes den i SRAM og Byte\_Num inkremeres med en. Hvis det derimod ikke er en kendt type/kommando, så sættes Byte\_Numb til en, og derved smides telegrammet væk, hvis ikke det er af en type, som der er implementeret i microcontrolleren. Den første implementering af kommunikations protokollen indeholdt ikke dette check for om det var en godkendt type allerede i modtagelsen. Dette gjorde at hvis der på en eller anden måde fik sneget sig en byte ind imellem modtagelsen af type, kommando og parameter. Så ville microcontrolleren skulle modtage 3 bytes før fejlen bliver opdaget. Dette resulterede i at efterfølgende telegrammer, ville være forskubbet med en byte og konsekvent ville fejle. Med den nuværende implementering vil microcontrolleren opdage fejlen med det samme, og efterfølgende telegrammer vil blive opfanget korrekt. Grunden til reimplementeringen var, at når man sendte telegrammer hurtigere end microcontrolleren kunne håndterer dem så, begyndte dens rx-buffer at blive overskrevet, og dette fik meget hurtigt ødelagt kommunikationen pga. forskubbede telegrammer. Den nuværende implementering lider ikke under dette, og vil blot ikke svarer på en kommando, hvis den oplever et overflow, men svarer efterfølgende igen, når dens buffer ikke oplever et overflow længere. Dette gør protokollen meget mere solid.

\begin{figure}[p]

	\centering
		\includegraphics[scale=0.8]{Billeder/Comm_Modtagelse.png}
	\caption{Flowchart over kommunikations protokollen ved modtagelsen af en byte)}
	\label{fig:Comm_Modtagelse}
	
\end{figure}

\paragraph{Eksekvering} 
\textit{Figur \ref{fig:Comm_Eksekvering} Beskriver eksekveringen som et flowchart.} \\
Eksekveringen bliver kørt når byte nummer 3, parametret, bliver modtaget. Her bliver typen og kommandoen igen tjekket for at finde ud af, hvilken kommando der skal eksekveres. Hvis der af en eller anden grund er en fejl på de SRAM addresser der indeholder typen og kommandoen. Så vil microcontrolleren ikke genkende det gemte telegram og selv resette igen, Så kommunikationen kan fortsætte bagefter, uden at den hænger eller forskyder telegrammerne. Efter eksekveringen af kommandoen er endt så vil Byte\_Num blive sat til en igen, og det hele vil begynde forfra.

\begin{figure}[p]

	\centering
		\includegraphics[scale=0.8]{Billeder/Comm_Eksekvering.png}
	\caption{Flowchart over kommunikations protokollen ved eksekveringen af en byte)}
	\label{fig:Comm_Eksekvering}
	
\end{figure}

\subsection{Tachometer}

Til at behandle signalet fra hall sensoren, benyttes input capture funktionen til Timer 1. Input capture funktionen har sit eget interrupt og fungerer ved automatisk at læse værdien af Timer 1's timer/counter registre (TCNT1A/TCNT1B) over i to input capture registre (ICR1L/ICR1H), når der kommer et interrupt. Input capture er den sekundære funktion til PD6, og sættes op i Timer 1's kontrol registre (TCCR1A/TCCR1B). Vi valgte at sætte den op med en prescaler på 1/8 og en falling edge interrupt trigger. Det giver en opdateringsfrekvens på 0.5 $\mu s$ og en maksimal tid på 32.5 ms, inden timer/counter registrene overflower. Det er fint nok, til at kunne rumme tiden imellem flere interrupts. I interruptrutinen beregnes tiden imellem 4 interrupts for at finde ud af, hvor lang tid motoren er om at dreje en omgang (Der sidder tre poler på motoren, så 4 interrupts vil svare til en hel omgang for motoren). Denne tid kan så bruges som et udtryk for hastigheden, da de to ting er omvendt proportionale. Bilens reelle hastighed kan også beregnes, men på grund af proportionaliteten er det ligegyldigt fra microcontrolleren's synspunkt, om man bruger det ene eller det andet - man skal bare huske at en høj hastighed, vil give en lav tid og omvendt, når man skriver programmet, der skal bearbejde hastigheden. Jo færre cycles man bruger på at få noget brugbart ud i den anden ende jo bedre i næsten alle tilfælde, når der også skal laves andre ting sideløbende med input capture interruptet.\\
For at beregne tiden skal der holdes styr på pulserne - der skal bruges en tid fra den første puls og en tid fra den sidste puls. Det letteste er at tælle en variabel op, hver gang der modtages et interrupt således at, når variablen er 1 så læser man den første tid, og når variablen er 4 så læses den sidste tid. Derefter trækkes de to tider fra hinanden for at finde forskellen. Alle værdier og variable gemmes i SRAM-hukommelsen på tildelte addresser, indtil de skal bruges igen.\\

Det viste sig, at hall sensoren gav mange udslag, når bilen holdt stille, og en af polerne var lige på grænsen til at passere sensoren. Vi antager, at det er, fordi feltet er meget ustabilt i grænseområdet. 


\subsection{Lap sensor}

For at initialisere microcontrolleren's komparator skal den sættes op i dens eget status register (ACSR) og i Special Function IO Registret (SFIOR). Man kan vælge flere forskellige inputs til komparatoren - PB2(AIN1) og PB3(AIN0) er henholdsvis ikke-inverterende og inverterende inputs som standard. Alternativt kan hele PORTA bruges til det inverterende og den interne bandgap reference på 2.56 V til det ikke-inverterende.  ACME bit'en i SFIOR enabler ADC multiplexeren, så den styrer hvilket ben på PORTA, der bliver brugt. ACBG bit'en enabler bandgap referencen.

\begin{figure}[h]

	\centering
		\includegraphics[scale=0.3]{Billeder/Table21-1.jpg}
	\caption{I tabellen kan man se, at ACME bit'en lader multiplexeren vælge hvilket ben på PORTA, der er input til komparatorens inverterende ben. Medmindre ADC'en er slået til - så bliver inputtet taget fra PB2(AIN1)}
	\label{fig:ACME}
	
\end{figure}

Man har også mulighed for at vælge, hvornår komparatoren skal sende et interrupt. Interruptet bliver sendt som en reaktion på, hvad der sker på komparatorens output-ben. Vi har sat den op til at sende et interrupt, når outputtet toggler tilstand, men der er også mulighed for at køre på enten falling eller rising edge. Det betyder i vores tilfælde ikke noget, hvilken metode man vælger på grund af måden koden bearbejder interruptet, men mere om det senere.\\

I interruptrutinen til lap sensoren beregnes omgangstiden. Hver gang målstregen krydses læses tiden fra tre SRAM addresser, hvor tiden microcontrolleren har været tændt, (i millisekunder) er gemt i 24 bits. Interruptrutinen sørger for at gemme et 24 bits time stamp i SRAM'en, hver gang den køres. Med den totale tid og et time stamp fra det forrige interrupt, er det bare et spørgsmål om at trække de to fra hinanden for at finde omgangstiden. For at forhindre ugyldige omgangstider, hvis der kommer mere end et interrupt på samme målstreg, kontrolleres det om omgangstiden er længere end 512 millisekunder se figur \ref{fig:LapTime}.

\begin{figure}[h]

	\centering
		\includegraphics[scale=0.5]{Billeder/LapTime.jpg}
	\caption{Her er den nuværende tid gemt i R0, R1, R2 og time stampet i R3, R4, R5. Ved at kontrollere, om den mellemste byte i R1 er under 2, kan det afgøres, om der er gået nok tid siden sidste time stamp, til at anse interruptet som gyldigt.}
	\label{fig:LapTime}
	
\end{figure}

Det er også efter denne kontrol, at man kan ændre på alle de parametre, der kun skal ændres en gang når målstregen krydses. Inden man forlader interruptrutinen, er det vigtigt at resette comparator interrupt flaget i ACSR. Hvis flaget er sat, når det globale interrupt bliver enablet efter rutinen, anser microcontrolleren det som om, den har modtaget et nyt interrupt, og så vil den køre hele rutinen igen.

\subsection{Hastighedskontrol}
\label{sec:Hastighedskontrol}

For at designe en hastighedskontrol, der selv regulerer PWM og bremser alt efter hvor langt, den målte hastighed er fra den ønskede, valgte vi at designe en progressiv kontrol, hvorved jo større forskellen imellem den ønskede og målte hastighed er, jo kraftigere vil microcontrolleren forsøge at justere. Tanken er at bruge den forskel som et direkte input til OCR2 eller bremsen, således at jo tættere bilen kommer på den ønskede hastighed jo lavere bliver værdien i OCR2, og lige sådan omvendt. Det medfører at det er umuligt for bilen, at stabilisere sig selv på den ønskede hastighed, men det gør ikke det store, da den ønskede hastighed er i forvejen en arbitrær værdi, som man skal eksperimentere sig lidt frem til. Det er både fordi, det ikke kan vides på forhånd, hvilke hastigheder der er ideelle i alle situationer, men også fordi den værdi microcontrolleren justerer efter, ikke er en rigtig hastighed, men tiden pr. motoromdrejning. Hele forløbet kan følges på flow chartet på figur \ref{fig:Speed}\\
Hastighedskontrollen er implementeret som en call-funktion, der bliver kaldt hver gang, hastigheden bliver opdateret i input capture interruptrutinen. Det har den ulempe, at hvis bilen går i stå, af en eller anden grund, så der ikke kommer flere interrupts, så kan den ikke starte igen af sig selv, hvis værdien i OCR2 er 0. \\
\subsubsection{PWM}
Det første der sker er at den ønskede, og den målte hastighed læses ind fra SRAM'en, og derefter findes forskellen mellem dem. For at afgøre om bilen kører for hurtigt eller for langsomt, kontrolleres fortegnet på den udregnede forskel - det kan gøres nemt ved at kigge på N-flaget i statusregistret. Hvis bilen kører for langsomt er forskellen negativ, og for at kunne arbejde med det bruges tallets to-komplement. Hvis forskellen er større end en på forhånd bestemt størrelse, så sættes OCR2 til 255 så duty cycle er på 100 procent. Hvis den er under læses forskellen direkte ind i OCR2. Eller det ville vi gøre, hvis hastigheden var en 8-bits værdi - vi gemmer hastigheden som en 16 bits størrelse, så derfor er vi nødt til at dividere tallet ned, så det passer i et 8 bits register. Så længe man bibeholder proportionaliteten, gør det ikke noget, at tallet bliver divideret ned. For at spare på cycles i en interruptrutine, der bliver kaldt forholdsvis ofte, divideres der kun med potenser af 2 - det kan laves nemmere ved at bruge logic shifts eller rotations (det svarer til at flytte kommaet, når man ganger eller dividerer med 10 i base-10 talsystemer). Ved at dividere tallet ned giver det også muligheden for nemt at sætte et maksimum på værdien, der kan læses ind i OCR2, når forskellen bliver brugt direkte. Hvis man eksempelvis sætter max-forskellen til 5760, så kan der højest læses 180 ind i OCR2, inden microcontrolleren regulerer med fuld kraft, idet $180 \cdot 32 = 5760$. Jo større tal man dividerer med, jo blødere reguleres motorkraften og jo større skal forskellen også være, for at der gives fuld kraft på motoren - vi fandt en udemærket balance med en skalering på 32.\\


\begin{figure}[h]

	\centering
		\includegraphics[scale=0.4]{Billeder/Braking.jpg}
	\caption{Ned til den nedre grænse bruge forskellen direkte i OCR2. I soft braking zonen læses 0 ind i OCR2 - derfra og op til den øvre grænse bruges bremsen i en tid bestemt af hastighedsforskellen. Hvis forskellen er større end de to grænser, bremses der eller gasses op for fuld kraft.}
	\label{fig:Forgiveness}
\end{figure}

\begin{figure}[hpt]

	\centering
		\includegraphics[scale=0.4]{Billeder/Speed.jpg}
	\caption{Her ses flow chartet til hele hastighedskontrollen. Med buffer menes grænsen imellem soft braking zonen og området hvor forskellen bliver brugt direkte som delay.}
	\label{fig:Speed}
	
\end{figure}


\subsubsection{Bremse}
Bremsen fungerer efter samme princip, men den regulerer bare bremsen med en tid i millisekunder. Når forskellen bliver positiv, læses der som standard 0 ind i OCR2 så duty cycle også bliver 0 - der sættes også en bit på en SRAM-addresse, som fortæller, at bilen er igang med at bremse. Helt i starten af hastighedskontrollen checkes denne bit, og hvis den er sat, springer program counteren til en anden funktion, der holder øje med, hvor lang tid der har været bremset i. Funktionen fungerer meget på samme måde som, den der holder øje med omgangstider - der sættes et time stamp, lige når bremsen bliver sat til, og så holder microcontrolleren øje med om der forløbet nok tid, til at den kan slå bremsen fra igen. Tiden bremsen er sat til i, bliver på samme måde som med PWM, når bilen kører for langsomt, ved at bruge hastighedsforskellen som direkte parameter. Den maksimale tid der bremses i med er 255 ms - det viste sig under tests at en bremsetid på mere end 255 ms, ikke har ret meget effekt. Samtidig kan man genbruge 8 bits formatet fra tidligere. Bilen finder selv ud af om der er brug for at bremse mere, efter de 255 ms er gået. Der er indbygget en lille buffer, så bremsen ikke bliver brugt, lige så snart forskellen bliver positiv - op til en vis grænse er der en slags soft braking zone se figur \ref{fig:Forgiveness}, hvor OCR2 bare er sat til 0. 
