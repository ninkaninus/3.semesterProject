\newpage
\section{Project Management}

Når et projekt når en vis størrelse, bliver det nødvendigt med projekt styring. Til dette formål er der forsøgt med en agil tilgang.\\

Der er blevet benyttet flere agile metoder.

\subsection{Scrum}

Scrum er en agil møde og organiserings metode, der er beregnet til små grupper. Et normalt møde er delt op i flere punkter og hver medlem af holdet, gennemgår på tur alle punkter.

\textbf{Mødepunkter}
\begin{itemize}
	\item Hvad har du lavet siden sidste scrum møde.
	\item Hvad vil du lave til næste gang?
	\item Er der nogle åbenlyse ting der forhindrer dig i at nå vores mål?
	\item Er der nogle nye ukendte opgaver der skal tilføjes til Sprint Backlog?
	\item Har du lært noget nyt der kan være brugbart for de andre gruppemedlemmer?
\end{itemize}

Ideelt set holdes mødet på samme tid hver dag, også selv om der skulle mangle nogle fra holdet.
Det er forsøgt at begrænse mødet til 20 min, andre ting som skal tages op, bliver noteret så det kan blive diskuteret efter mødet, Dette resulterer i at mødet kan holdes kort og præcis, så det ikke tager for meget arbejdstid fra folk, der ikke skal bruge det til noget.\\

Under hvert møde, blev der lavet et mødereferat, hvor hver persons svar til de forskellige punkter, blev skrevet ind.\\ % inset henvisning til billag: Scrum møde referat.



\textbf{Effekt}\\

Hele holdet får en god idé om hvor projektet er på vej hen, samt hvad de øvrige medlemmer laver.
Eventuelle problemer bliver bragt frem, og ny viden der er relevant kan blive delt med alle.\\
Det giver et godt overblik og sammenhæng mellem hvad der bliver arbejdet på. Der har dog under projektet nok været brugt for meget tid på Scrum møder i forhold til den gavnlige effekt det har givet.


\subsection{Pair programming}

Pair programming går som navnet antyder, ud på at to personer programmerer sammen på en computer. Det kan virke som et spild af arbejdskraft, da der jo kun kan være en der koder af gangen, men det kan hjælpe til at give noget bedre kode. Grunden er at der hele tiden vil være en til at se hvis der bliver laver fejl, samt komme med gode ideer. Det er også en god måde at dele erfaring og viden om projektet.\\

Det er delvis blevet benyttet, men der har også været en del individuel programering.

\textbf {Effekt}
Det har givet en bredere og bedre forståelse for koden, og har forhindret nogle fejl som måske ellers først ville blive fanget senere.



\subsection{Iterativ og Incrementiel udvikling}
Der har været mange metoder til at strukturere, udviklingen af et program som f.eks. "Vandfalds metoden". I den metode, laver man hele programmet ud i et, fra start til slut og der bliver ikke gået tilbage for at lave noget om. Det er ikke en fleksibel måde at udvikle et program på, hvor det vil være meget svært at tilpasse sig ændringer i kravspecifikationerne. Det kræver også at der er fuldstændig styr på hvad der skal laves og hvordan.\\

For at undgå de svagheder, er der til dette projekt valgt en iterativ og incrementiel tilgang.\\
Hele projektet er delt op i flere dele. Hver del er gradvis udviklet og ændringer i en del har en effekt på de andre. Der er lavet test løbende og delene er blevet tilpasset de nye erfaringer der er blevet gjort.


\textbf{Effekt}

Da der løbende har været test af de enkelte dele og hvordan de spiller sammen, har det været muligt at lave ændringer efterhånden som det har været nødvendigt.\\

%Er Tidsplan delen nødvengig

\subsection{Tidsplan}
For at holde styr på hvad der skulle laves hvornår og hvor meget der manglede, blev der lavet en tidsplan. Denne blev drøftet ved de fleste møder, og opdateret efterhånden som projektet skred frem.

\textbf{Effekt}
En dynamisk tidsplan, giver et godt overblik over hvor projektet er på vej hen, samtidig med at der kan tages hensyn til de ændringer som skulle opstå.?????


