\newpage
\section{Project Management}

Når et projekt når en vis størrelse, bliver det nødvendigt med projekt styring. Til dette formår er der forsøgt med en agile tilgang.\\


\subsection{Scrum}

Scrum er en agile møde og organiserings metode, der er beregnet til små grupper. Et normalt møde er delt op i flere dele og hver medlem af holdet, gennemgår på tur alle delene.

\textbf{Mødepunkter}
\begin{itemize}
	\item Hvad har folk lavet siden sidste scrum
	\item Hvad vil vi lave til næste gang?
	\item Er der nogle åbenlyse ting der forhindrer os i at nå vores mål?
	\item Er der nogle nye ukendte opgaver der skal tilføjes til Sprint Backlog?
	\item Har du lært noget nyt der kan være brugbart for de andre gruppemedlemmer?
\end{itemize}

Mødet holdes ideelt på samme tid, også selv om der skulle mangle nogle fra holdet. Det er forsøgt begrænset til 20 min og ting der skal snakkes om, bliver noteret så det kan blive diskuteret efter mødet.
Det gør at mødet kan holdes kort og præcis, så det ikke tager for meget af arbejds fra folk, der ikke skal bruge det til noget.\\

Under hvert møde, blev der lavet et mødereferat, hvor hver persons svar til de forskellige punkter, blev skrevet ind.




\textbf{Effekt}\\

Hele holdet får en god ide, om hvor projektet er på vej hen og hvad de andre laver. Eventuelle problemer bliver bragt frem, og ny viden der er relevant kan blive delt med alle.\\
Det giver et godt overblik og sammenhæng mellem hvad der bliver arbejdet på.


\subsection{Iterativ og Incremental udvikling}
Der har været mange metoder til at strukture, udviklingen af et program som f.eks. "Vandfalds metoden". I den metode, laver man hele programmet ud i et, fra start til slut og der bliver ikke gået tilbage for at lave noget om. Det er en meget ufleksibel måde at udvikle et program på, hvor det vil være meget svært at tilpasse sig ændringer i kravspecifikationerne. Det kræver også at der er fuldstændig styr på hvad der skal laves og hvordan.\\

For at undgå de svagheder, er der til dette projekt valgt, en iterativ og incremental tilgang.\\
Hele projektet er delt op i flere dele. Hver del er gradvis udviklet og ændringer i en del har en effekt på de andre. Der er lavet test løbende og delene er blevet tilpasset de nye erfaringer der er blevet gjort.


\textbf{Effekt}

Da der løbende har været test af de enkelte dele og hvordan de spiller sammen, har det været muligt at lave ændringer efterhånden som det har været nødvendigt. 



\subsection{Kravspecifikation}

TekstTekstTekstTekstTekst\\
TekstTekst\\
TekstTekstTekst \footnote{Test footnote}\\
Tekst

\begin{itemize}
	\item Tekst
	\item Tekst
	\item Tekst

\end{itemize}

%Signatur til matematik
\[(5)+5=10\]


$\displaystyle \sum$


$\displaystyle \int x^5+b_1$


$\frac{n!}{k!(n-k)!} = \binom{n}{k}$

\subsection{Projektafgrænsning}

Tekst

\begin{itemize}
	\item Tekst
	\item Tekst
	\item Tekst
\end{itemize}

Tekst


\subsection{Tidsplan}