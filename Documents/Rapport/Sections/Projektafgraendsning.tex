\section{Projektafgrænsning}
I forbindelse med dette projekt vil der blive udarbejdet en distribueret applikation i form af en chat-klient. Denne klient skal understøtte de mest basale funktionaliteter, der vil kunne forventes heraf. Dette indebærer afsendelse og modtagelse af tekstbeskeder, men skal kunne udbygges til at sende andre former for beskeder (eks. lyd- eller billedefiler). Denne besked skal transmitteres gennem brug af DTMF-toner. 
Projektet vil primært fokusere på selve transmissionen af data, og hvordan dette kan foregå på en generel, effektiv og pålidelig måde ved brug af DTMF-toner. Opbygningen af dette system vil derfor tage udgangspunkt i Internettet, da dette system understøtter kommunikation og stabilitet i en form, relevant for dette projekt. 



\subsection{Arbejds- og fagområder}

I og med at projekt strækker sig over et helt semester, skabes der en mulighed for at opnå et tværfagligt sammenspil mellem semesterets fire fagligheder. Disse fag er \textit{Digital signalbehandling} (DSP), \textit{ C++ programmering}, \textit{Datakommunikation} og \textit{Software udvikling}. Der blev derfor udarbejdet en liste over disse fagområder, kombineret med de arbejdsopgaver disse blandt andet forventes at indebære.   


\subsubsection{Software}

\begin{itemize}[noitemsep]
  \item Digital signalbehandling
  \begin{itemize}[noitemsep]
    \item Filtrering af signalet for optimal transmission
    \item Problematikker i overgangsfasen mellem DTMF-toner
  \end{itemize}
  \item Datakommunikation
  \begin{itemize}[noitemsep]
    \item Protokoller
    \item Flow- og fejlkontrol
    \item Adressering i forbindelse med flere modtagere
    \item Kollision af datapakker skal undgås
  \end{itemize}
  \item C++
    \begin{itemize}[noitemsep]
    \item Mulighed for overvågning af transmissionen i forbindelse med debugging
    \item Objekt-orienteret programmering
    \item Dynamisk tilpasning af signal ift. afstand og vinkel
    \item Udvikling af brugerinterface til chat-klient
  \end{itemize}
\end{itemize}

\textbf{Softwareudvikling}\\
Projektet kommer til at have ekstraordinært fokus på project management, i form af modellen Scrum. Arbejdsforløbet vil blandt andet inddeles i mindre ”Sprints”, og gruppen vil arbejde tæt sammen, med fokus på en høj grad af vidensdeling.

\subsubsection{Hardware}

\begin{itemize}[noitemsep]
  \item Signalstyrkens påvirkning på datatransmission.
  \item SNR (tolerancen, målinger og beregninger)
  \begin{itemize}[noitemsep]
  \item Hvordan opnås en hurtig og effektiv transmission.
  \item Overvejelser omkring båndbredde.
  \end{itemize}
  \item Genklangs inflydelse på transmissionen
  \item Vinkling og afstand
  \begin{itemize}[noitemsep]
  \item Dynamisk og statisk placering af hardware
  \end{itemize}
  \item Ændring af transportmedie, eksempelvis ved forhindringer mellem sender og modtager.
\end{itemize}
