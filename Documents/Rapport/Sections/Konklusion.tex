\section{Konklusion}
Projektets primære målsætninger er alle blevet nået.
Det kan lade sig gøre at transmittere data via DTMF toner, samt at etablere en rimelige stabil og fejlfri forbindelse. Der kan sendes meget information ved at bruge flere frekvenser oven i hinanden.\\
Lyd er dog ikke et godt medie, til transmission af data. Det har en lav udbredelses hastighed, det er normalt et meget støjfyldt medie og signalet bliver nemt forvrænget.

En simple protokol med lidt overhead, er nok den bedste man kan bruge, da det tager lang tid at sende data. %Udkast. Der skal nok skrives noget mere



%Afstanden mellem højtaleren og mikrofon har meget at sige, i forhold til 


%Virker mere som en opsamling på hvad der er lavet og ikke en konklusion
Et program er udviklet, som implementerer en 1-til-1 chat-klient, der benytter sig af DTMF toner til datatransmission. Programmet er udviklet med en lagdelt klassestruktur, som er inspireret af internettets protokoller. Der kan afsendes og modtages DTMF-toner ved lav afstand. En protokol til denne dataoverførsel er udarbejdet, som blandt andet opdeler data i pakker og foretager flow- og fejlkontrol. Således forhindres fejlagtig data i at nå de højere lag.






%----------------------------------------------%


%Jeg syndes ikke det afsnit her er relevant SBV.

XXXX skal der tilføjes en konklusion på sekundære målsætninger?
\subsection{Sekundære opgaver}
\begin{itemize}[noitemsep]
\item Systemet kan udbygges til at tage højde for vinkling og afstand mellem afsender og modtager, ved dynamisk og statisk placering
\item Undersøgelse af genklangs indflydelse på signalet
\item Chatten skal kunne sende flere forskellige datatyper, eksempelvis lyd eller billeder
\item I tilfælde af en videreudvikling der tillader flere modtagere, skal der arbejdes med adressering i protokollen
\item Optimering af transmissionshastighed
\item Der kan laves beregninger på systemets SNR (tolerancen, målinger og beregninger)
\end{itemize}