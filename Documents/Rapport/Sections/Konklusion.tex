\section{Konklusion}
Det kan konkluderes, at DTMF-toner kan genereres og transmitteres igennem C++'s SFML-bibliotek. Disse toner kan modtages ved hjælp af Goertzel-algoritmen, som bliver benyttet til at analysere og lede efter de kendte frekvenser i DTMF-tonerne. Det er igennem undersøgelser blevet konkluderet, at det er svært at forudsige, hvordan frekvensernes magnitude ændrer sig, ved forskellige afstande og vinkler. Derfor kan det være svært at præ-programmere en grænseværdi til at finde frekvenserne. En anden konklusion har været, at en kommunikationsprotokol kan designes og implementeres som et lagdelt stykke software, der strukturelt læner sig op ad internettet. Ansvaret kan fordeles til C++-klasser, der alle imiterer forskellige af internettets lag. Der konkluderes også, at transmissionshastighed og stabilitet blandt andet kan optimeres igennem fejl- og flowkontrolmekanismer, som garanterer korrekthed og minimerer spild i dataoverførslen. Til afslutning kan det konkluderes, at hele den designede kommunikationsprotokol kan benyttes på adskillige meningsfyldte måder. Ét designet applikationslag tillader to brugere at have en tekst-baseret chatsession. Grundet dataformatet i den lagdelte struktur og den stabile transmission, vil det imidlertid være trivielt at udbygge programmet med andre, meningsfyldte funktioner.